\documentclass{article}

\usepackage{enumitem}

\begin{document}

\begin{center}
\LARGE Guide Python du physicien
\end{center}

\section*{Objectif}
Le but de ces dépannages est de montrer de nombreux outils en Python en sciences. Autant que possible, ce petit cours va suivre le motto de Arch Linux \textbf{KISS}: \textit{Keep it simple, stupid}. 

Les présentations se feront dans une classe et vont durer moins de 1h (je vise 30 min). Je vais essayer d'en faire une à chaque semaine et de distribuer un document \LaTeX{} qui résume le contenu.

\section*{Sujets proposés}

\begin{enumerate}
    \item Modifier les variables d'environnement, installer des librairies avec pip 
    \item Faire différents graphiques avec matplotlib, personnaliser les figures
    \item Sauvegarder, charger et manipuler des fichiers de données 
    \item Utiliser numpy efficacement
    \item Introduction à la programmation orientée objet (c'est quoi une classe)
    \item Suite de la programmation orientée objet(héritage, méthodes statiques)
    \item Initation à git
    \item Utiliser \textit{sys}, \textit{os} et autres outils pour les chemins
    \item \textit{Curve fitting}, \textit{fsolve} et intégrales numériques
    \item Principes de programmation de base
    \item Divers outils de Python (\textit{map, zip, enumerate, filter, iter})
    \item Divers outils de Python 2(\textit{argparse}, types de copies, fonctions génératrices)
    \item Compiler \LaTeX{} localement (ne plus dépendre de Overleaf)
    \item vim (éditeur de texte puissant doté de nombreux outils)
\end{enumerate}

\end{document}
